%\chapter{L'École Centrale de Nantes et le campus universitaire}
\chapter{The Ecole Centrale de Nantes and its campus}

%\section{L'enseignement supérieur français}\trad
\section{French higher education}

%Comme pour les autres pays européens, l'enseignement supérieur est basé sur le système « LMD » : Licence Master Doctorat.
%En France, il n'y a pas que les universités, mais il existe également des Grandes Écoles, dites écoles d'ingénieurs, comme l'École Centrale Nantes.
%Pour débuter un doctorat en France, on peut avoir suivi soit un cursus universitaire (en passant donc par un master recherche) soit venir d'une école d'ingénieurs.

As in other European countries, higher education is based on the "LMD": Bachelor, Masters, Doctorate.
In France, there are not only universities but Grandes Ecoles, known for their engineering divisions, such as Ecole Centrale Nantes.
To start a PhD in France, you must either have a master degree (thus passing by a research master) or an engineering diploma.

\section{All you ever wanted to know about the École Centrale de Nantes}
\subsection{Several figures}
The school was founded more than 90 years ago.
Originally called the IPO (Institut Polytechnique de l'Ouest since 1919), then ENSM (École Nationale de Mécanique since 1947), and finally ECN (École Centrale de Nantes since 1991), it is part of the group of Centrale Schools in 1991.
It was originally based in the city center, and was then relocated on the current location in 1977.

%L'école a été fondée il y a plus de 90 ans.
%D'abord appelée l'IPO (Institut Polytechnique de l'Ouest depuis 1919), puis l'ENSM (École Nationale de Mécanique à partir de 1947), elle est rattachée au groupe des Écoles Centrales en 1991.
%Elle était initialement basée en centre ville, elle est arrivée sur le campus du Tertre en 1977.
\todo{Vérifier ces chiffres}
\paragraph{People}
\begin{itemize}
  \item 1850 students (1200 engineering students, 200 Master students, 200 PhD students),
  \item 550 researchers, associate professors and engineers,
  \item 150 members of the technical and administrative staff.
\end{itemize}
\paragraph{The place}
\begin{itemize}
  \item 40000 m\textsuperscript{2} of buildings,
  \item a campus of 16 ha,
  \item 5 laboratories labelled by the CNRS (the French national institute of research).
\end{itemize}
\paragraph{Centrale Nantes in France and in the world}
\begin{itemize}
  \item 12\textsuperscript{th} best engineering school of France (2012 ranking),
  \item Partnerships with 115 universities in 40 different countries.
\end{itemize}

\subsection{The most important people of the school}
Here is a non-exhaustive list of important people that you will probably meet during your 3 years of thesis.
%Voici une petite liste (non exhaustive) de personnalités importantes du campus, que vous serez amenés à croiser durant vos 3 ans de thèse.

\newcommand{\largcolp}{0.2\textwidth}
\newcommand{\largcol}{0.7\textwidth}

\newcommand{\pers}[4]{%
\begin{tabular}{m{\largcolp}}
\\
\ifthenelse{\equal{#1}{}}{~\vspace*{2.2cm}}{\raisebox{-.5\height}{\includegraphics[width=2.5cm]{images/#1}}}
\\~
\end{tabular} &
\begin{tabular}{m{\largcol}}
\textbf{#2} \\ \textit{(#3)} \\ #4
\end{tabular}\\\hline}

\noindent
\begin{tabular}{cm{\largcol}}
  \pers{poitou}{Arnaud \textsc{Poitou}}{Director}{He's everybody's boss. We all have to swear allegiance to him, even if you not often have to be dealing with him}%C'est notre chef à tous. On lui doit allégeance, même si en pratique, on a rarement affaire à lui.}&
%  \pers{}{}{Dean of study}{Comme son statut l'indique, il est davantage en lien avec les élèves. Amis moniteurs, si vous avez des problèmes avec vos enseignements, c'est peut-être à sa porte qu'il faut aller frapper.}&
  \pers{hascoet}{Jean-Yves \textsc{Hascoet}}{Research director}{As research people, he's our boss. He manages the research of the school (projets, grants, activities, \dots) and he may have to look after you in case of [very big] issues with your thesis.}%C'est notre boss à nous. Il gère la politique de recherche de l'école (les projets, l'argent, les diverses activités, \dots) et peut s'occuper de nous en cas de [gros] problèmes avec notre thèse.}\\
%  \pers{}{Olivier \textsc{Menard}}{Main secretary}{He deals with all administrative matters that may concern you (all the enrolment red tape, do you remember?). He know all official texts, so he will be able to answer all your administrative concerns.}%Il s'occupe des grandes décisions administratives qui nous concernent (oui, la paperasse de l'inscription, c'est en partie pour son service\dots). Véritable encyclopédie des textes officiels, il pourra répondre aux questions auxquelles tout autre secrétaire aura jeté l'éponge.}&
%  \pers{}{}{Directeur de la communication}{Lui et son équipe gèrent la diffusion des informations et l'événementiel sur le campus. Son travail dépend notamment de votre participation, alors n'hésitez pas à l'informer de tout ce qu'il se passe dans votre labo.}&
%  \pers{}{}{Directeur adjoint}{C'est l'homme de terrain. Il met à exécution les grandes décisions « concrètes » de l'administration. Vous recevrez notamment pas mal de mails durant l'année sur d'éventuels travaux, changement d'organisation, etc. de sa part.}
  \pers{boutin}{Édith \textsc{Boutin}}{PhD students secretary}{She deals more specifically with PhD students. You will see her again for re-enrolment each year, and many other enquiries.}
  \pers{ta}{Your supervisors}{Your supervisors}{They will depend of many things, especially your research subject. They will guide you through your PhD years\dots If they have spare time for you! Indeed, you will be looking for them often, really often.}
\end{tabular}

\subsection{Students}
There are many different kind of students in the school:
%L'école renferme une population d'étudiants assez hétéroclite. Il y a :
\paragraph{Engineering students (Ei1, Ei2, Ei3)} They enroll for 3 years after a competitive examination at the end of French \textit{classes préparatoires}, and they make up the biggest part of the Centrale Nantes population. They drive the community life of the school all year long. %Ils sont admis à l'école sur concours après une classe prépa (bac+2) et constituent la majorité des étudiants du campus et sont présents pour 3 ans. Très actifs sur le plan associatif, ils animeront le campus tout au long de l'année. Les premières années vivent dans la résidence près du campus. Depuis peu, certains élèves ingénieurs travaillent en alternance au même titre que les ITII (voir plus bas).
\paragraph{Master students (M1 \& M2)} They come from several places around the world, of other engineering school and universities than Centrale Nantes, but they share some courses with the engineering students. %Originaires d'un parcours universitaire tiers, les masters partagent souvent leurs cours avec les élèves ingénieurs. On peut notamment citer les Master Erasmus-Mundus attirant des étudiants du monde entier!
%\paragraph{Les ingénieurs par alternance ITII Pays de la Loire (1ère, 2ème et 3ème année)} Il s'agit d'étudiants en formation par alternance (2 semaines en cours, 2 semaines en entreprise). L'ITII Pays de la Loire est un organisme de formation extérieur à l'École Centrale Nantes, cependant les cours se font sur le campus. Officiellement, ce ne sont pas des étudiants. Dans la pratique, ils sont exactement comme les autres, à la différence qu'ils ne sont présents que 2 semaines sur 4.
\paragraph{PhD students (D1, D2 \& D3)} That's you!

\subsection{The labs}
There are 5 laboratories on the campus:
%Le campus de l'École Centrale comprend 5 labos (voir le plan du campus) :
\paragraph{The GeM} Its real name is “Institut de Recherche en Génie civil et Mécanique”, and it deals with mechanics and civil engineering, like crash-tests, material studies, mechanical simulations~\dots %You can find it on two other places: the science university and in St Nazaire. %On y fait de la simulation mécanique, du génie civil, des études de matériaux, de la dynamique rapide (crash-test)\dots Parmi tous les thèmes de recherche (et il y en a beaucoup), deux ressortent : La méthode X-FEM (méthode des éléments finis enrichis) et la PGD (méthode de décomposition de variables pour résoudre des problèmes beaucoup plus rapidement). D'autre part, un domaine d'application particulièrement en vogue ces temps ci est celui des matériaux composites.
\paragraph{The LHEEA} The “Laboratoire de recherche en Hydrodynamique, Énergétique et Environnement Atmosphérique” owns several big experimentation pools, which allows to test model boats, and a wind tunnel. %C'est sans doute le laboratoire qui possède le matériel expérimental le plus impressionnant du campus : un bassin de carène (pour tracter des maquettes de bateaux), un bassin de houle (pour faire des vagues), une soufflerie\dots Le LHEEA est morcelé en plusieurs équipes dont les principaux thèmes sont l'hydrodynamique et génie océanique (projet « SEAREV » de récupération d'énergie des vagues), l'énergétique des moteurs à combustion interne, l'étude de la dynamique de l'atmosphère en milieu urbain.
\paragraph{The IRCCyN} This “Institut de Recherche en Communications et Cybernétique de Nantes” is mainly located in the big weird oval building. It gathers a dozen of different disciplines: robotics, computer science, real time, machining. %Principalement installé dans le bâtiment S (le bâtiment relativement moderne, de forme ovale), l'IRCCyN concentre douze thématiques de recherche très variées. On y travaille par exemple sur la commande, la robotique, la conception, la fabrication additive, l'optimisation multidisciplinaire, les systèmes temps réels et même\dots la psychologie ! L'IRCCyN est réparti sur 4 sites de Nantes : l'École Centrale, l'École des Mines, l'École Polytechnique et l'IUT.
\paragraph{The CERMA} Also known as “Centre de Recherche Méthodologique d'Architecture”, it depends of the school of architecture of Nantes. Its main field is architecture physics. %Chapeauté depuis peu par l'École Centrale, le CERMA est avant tout un laboratoire dépendant de l'École Nationale Supérieure d'Architecture de Nantes. On y travaille notamment sur toute la physique liée l'architecture et l'environnement urbain.
\paragraph{The Jean Leray mathematics laboratory} Only a small part of this laboratory is on the Centrale Nantes campus, while the other part lies in the university of Nantes. It deals with non-linear differential equations, which can have many applications. %On n'en entend pas trop parler, mais il est là, caché dans une partie du bâtiment E ! En fait, c'est parce que l'essentiel se trouve à la fac de sciences et seul un petit bout est à l'École Centrale Nantes. La recherche à Centrale se concentre sur l'étude des équations aux dérivées partielles non-linéaire avec des domaines d'application pouvant aller des problèmes dynamique de populations à l'étude des changement d'état de matériaux supraconducteurs.

%\subsection{Map of the École Centrale Nantes}
%\todo{Inclure le plan}


\section{Sport and culture on campus}
\subsection {Where do sports?}
Wanna romp? Out of your 10 m \ {2} textsuperscript office?
Will make the sport! Three structures allow you that.

\subsubsection{The Athletic Association Centrale Nantes (AS)}
It is the association managed by the Sports Office (BDS) students from Central. As students, doctoral students may enroll in AS.
\paragraph{Contacts}
\begin{itemize}
  \item The email address of the office of sports: \ texttt {} bds@ec-nantes.fr
  \item Bureau 1 \ {st} textsuperscript floor of L
  \item \ texttt {} http://bds.campus.ec-nantes.fr
\end {itemize}
\paragraph{How to apply}
\begin{itemize}
  \item A check \ EUR {55},
  \item A medical certificate \ footnote {-filled meadows certificates are available at the office of BDS} non-cons-indication (if you specify the competition)
\end{itemize}

\paragraph{Highlights}
\begin{itemize}
  \item [$+$] Many sports (17)
  \item [$+$] Possibility of academic competitions,
  \item [$+$] Diversity of levels,
  \item [$+$] Some offer sports training with teacher
  \item [$+$] You may find yourself playing with your own students!
\end {itemize}
\paragraph{Weaknesses}
\begin{itemize}
  \item [$-$] You may find yourself playing with your own students,
  \item [$-$] Sometimes too crowded,
  \item [$-$] Requires consistency in training.
\end{itemize}

\paragraph{Warning} The competitions are usually held on Thursday afternoon (often at the same time as seminars).
As drives, they require a special commitment vis-à-vis your team and your opponents.
Before you commit, although check your obligations doctoral allow you to regularly participate in these competitions.

\paragraph{Some events throughout the year}
\begin{itemize}
  \item Outside Central: The Inter-Central, Challenge Lyon, BER (Brest), CAS (Supélec Rennes), the TOSS (Supélec Paris).
  \item In Central: Tournament 4 balls (T4B) 3 rackets tournament (T3R), Inter-groups.
\end{itemize}

\subsubsection {The Association Sportive du Personnel (ASP)}
It is the staff association of the Ecole Centrale (teachers, administrators, technicians, etc.. PhD and!). The association offers only four sports (futsal, tennis, gym and badminton) but asks only change. If you are motivated to get a sport, especially not hesitate! FYI, before there was table tennis and dance.

\paragraph{} 12 Slots - 13:30.
\paragraph{Contacts}
\begin {itemize}
  \item Chairperson: \ texttt {jean-paul.bouganne @ ec-nantes.fr} (02 40 37 25 92)
  \item Intranet site of ECN
\end {itemize}
\paragraph{How to apply}
\begin{itemize}
  \item A check \ EUR {13} (+ \ EUR {13} for tennis \ footnote {The tennis requires to pay a license FFT}).
  \item A medical certificate of non-cons-indication.
\end{itemize}

\paragraph{Highlights}
\begin{itemize}
  \item Cheap,
  \item Often fewer people
  \item Ideal to go romp without headaches
\end{itemize}
\paragraph{} Weaknesses
\begin{itemize}
  \item Few sports (4),
  \item No competition possible (except for tennis!)
\end{itemize}

\subsubsection {The SUAPS}
SUAPS is the agency responsible for sports in college Nantes.
This is a BIG structure based in the gym in front of the restaurant U.
Even if you are not a student of the college, you can still sign up at SUAPS.
This is usually a good alternative if the sport you desire is not available on the campus of Central.

\paragraph{Warning} If, as a PhD student, you are not enrolled at the university, you will be considered "foreign student."
Registration will cost more (\ EUR {50} instead of \ EUR {35}), and some sports "popular" you are not allowed.
Visit the website for more information SUAPS.

\paragraph{Highlights}
\begin{itemize}
  \item Many sports (47), including sports "rare"
  \item Many hardware
  \item Large variety of levels,
  \item Provides slots for the pool.
\end{itemize}
\paragraph{Weaknesses}
\begin{itemize}
  \item Sometimes too crowded,
  \item Off campus,
  \item You can be considered "foreign student" in college, and therefore not a priority in some sports.
\end{itemize}

\subsection {Cultural Associations Student}
\subsubsection {At the Ecole Centrale}
\paragraph{AED} Association of Students in PhD from Ecole Centrale Nantes. It brings together all the docs on the site of the ECN and organizes events as festive as scientists. Join us on Facebook group "ECN-AED" to see the thread information. Website: \ url {} http://website.ec-nantes.fr/aed/
\paragraph{AMAP Ecole Central} BDS and Central Green ECN offer in September 2011 baskets of organic vegetables, delivered weekly. Get more information at the start of 2011-2012, or email.

On the other hand, the offices of students (BDE, BDA, BDS) organizing an evening in September retraction associations. You can discover all the clubs and eventually you register! Learn about the date.

\subsubsection{On campus}
\paragraph{Association "Feet in the head"} Provides film-philosophy, the aim of the association is to remove part of the philosophy of education and to open wider debates on such freedom. Site: \ url {} http://depidalate.free.fr/
\paragraph{House of Games} Every Wednesday noon at TU, you can discover and participate in games and wooden games.

\section{Sport et culture sur le campus}\trad
%\todo{Vérifier les infos de cette section}
%\subsection{Où faire du sport ?}\trad
%Envie de te défouler ? De sortir de tes 10 m\textsuperscript{2} de bureau ?
%Va faire du sport ! Trois structures te permettent cela.
%
%\subsubsection{L'Association sportive de Centrale Nantes (AS)}
%C'est l'association gérée par le Bureau Des Sports (BDS) des étudiants de Centrale. En tant qu'étudiants, les doctorants peuvent s'inscrire à l'AS.
%\paragraph{Contacts}
%\begin{itemize}
%  \item L'adresse courriel du bureau des sports : \texttt{bds@ec-nantes.fr}
%  \item Bureau au 1\textsuperscript{er} étage du bâtiment L
%  \item \texttt{http://bds.campus.ec-nantes.fr}
%\end{itemize}
%\paragraph{Modalités d'inscription}
%\begin{itemize}
%  \item Un chèque de \EUR{55},
%  \item Un certificat médical\footnote{Des certificats prés-remplis sont disponibles au bureau du BDS} de non-contre-indication (précisez si vous faites de la compétition)
%\end{itemize}
%
%\paragraph{Points forts}
%\begin{itemize}
%  \item[$+$] Beaucoup de sports (17),
%  \item[$+$] Possibilité de faire des compétitions universitaires,
%  \item[$+$] Diversité de niveaux,
%  \item[$+$] Certains sports proposent des entrainements avec professeur,
%  \item[$+$] Tu peux te retrouver à jouer avec tes propres élèves !
%\end{itemize}
%\paragraph{Points faibles}
%\begin{itemize}
%  \item[$-$] Tu peux te retrouver à jouer avec tes propres élèves,
%  \item[$-$] Parfois trop de monde,
%  \item[$-$] Nécessite une régularité dans les entraînements.
%\end{itemize}
%
%\paragraph{Attention} Les compétitions ont généralement lieu les jeudis après-midi (souvent en même temps que des séminaires).
%Tout comme les entraînements, elles demandent un engagement particulier vis-à-vis vos équipes et de vos adversaires.
%Avant de t'engager, vérifie bien que tes obligations de doctorants te permettent de participer régulièrement à ces compétitions.
%
%\paragraph{Quelques événements tout au long de l'année}
%\begin{itemize}
%  \item Hors de Centrale : Les Inter-Centrales, le Challenge Lyon, le TEB (Brest), les CAS (Supélec Rennes), le TOSS (Supélec Paris).
%  \item Dans Centrale : Tournoi des 4 ballons (T4B), tournoi des 3 raquettes (T3R), Inter-groupes.
%\end{itemize}
%
%\subsubsection{L'Association Sportive du Personnel (ASP)}
%C'est l'association du personnel de l'École Centrale (professeurs, administratifs, techniciens, etc. et doctorants !). L'association ne propose que quatre sports (foot en salle, Tennis, musculation et badminton) mais ne demande qu'à évoluer. Si tu es motivé pour monter un sport, n'hésite surtout pas ! Pour info, il y avait avant du ping-pong et de la danse.
%
%\paragraph{Créneaux horaires} 12h--13h30.
%\paragraph{Contacts}
%\begin{itemize}
%  \item Président : \texttt{jean-paul.bouganne@ec-nantes.fr} (02 40 37 25 92)
%  \item Intranet du site de l'ECN
%\end{itemize}
%\paragraph{Modalités d'inscription}
%\begin{itemize}
%  \item Un chèque de \EUR{13} (+ \EUR{13} pour le tennis\footnote{Le tennis nécessite de payer une licence FFT}).
%  \item Un certificat médical de non-contre-indication.
%\end{itemize}
%
%\paragraph{Points forts}
%\begin{itemize}
%  \item Pas cher,
%  \item Souvent moins de monde,
%  \item Idéal pour aller se défouler sans prise de tête
%\end{itemize}
%\paragraph{Points faibles}
%\begin{itemize}
%  \item Peu de sports (4),
%  \item Pas de compétition possible (sauf pour le tennis !)
%\end{itemize}
%
%\subsubsection{Le SUAPS}
%Le SUAPS est l'organisme responsable des sports à la fac de Nantes.
%C'est une GROSSE structure basée dans le gymnase en face du resto U.
%Même si tu n'es pas étudiant de la fac, tu peux quand même t'inscrire au SUAPS.
%C'est en général une bonne alternative si le sport que tu désires n'est pas disponible sur le campus de Centrale.
%
%\paragraph{Attention} Si, en tant que doctorant, tu n'es pas inscrit à l'université, tu seras considéré comme « étudiant extérieur ».
%L'inscription te coûtera plus cher (\EUR{50} au lieu de \EUR{35}), et certains sports « populaires » ne te seront pas autorisés.
%Rendez-vous sur le site du SUAPS pour plus d'informations.
%
%\paragraph{Points forts}
%\begin{itemize}
%  \item Beaucoup de sports (47), dont des sports « rares »,
%  \item Beaucoup de matériel,
%  \item Grande diversité de niveaux,
%  \item Propose des créneaux pour la piscine.
%\end{itemize}
%\paragraph{Points faibles}
%\begin{itemize}
%  \item Parfois trop de monde,
%  \item Hors du campus,
%  \item Tu peux être considéré comme « étudiant extérieur » à la fac, et donc non-prioritaire sur certains sports.
%\end{itemize}
%
%\subsection{Associations culturelles étudiantes}\trad
%\subsubsection{À l'École Centrale}
%\paragraph{AED} Association des Élèves en Doctorat de l'École Centrale Nantes. Elle fédère tous les doctorants présents sur le site de l'ECN et organise des événements tant festifs que scientifiques. Rejoins-nous sur le groupe Facebook « AED-ECN » pour voir le fil des infos. Site internet : \url{http://website.ec-nantes.fr/aed/}
%\paragraph{AMAP Ecole Centrale} Le BDS et Centrale Vert de l'ECN propose à la rentrée 2011 des paniers de légumes bio, distribués toutes les semaines. Tu auras de plus amples informations à la rentrée 2011-2012, ou par mail.
%
%D'autre part, les bureaux des élèves (BDE, BDA, BDS) organisent en Septembre une soirée de rentrée des associations. Tu pourras y découvrir tous les clubs et éventuellement t'y inscrire ! Renseigne-toi sur la date.
%
%\subsubsection{Sur le campus universitaire}
%\paragraph{Association « Des pieds dans la tête »} Propose des ciné-philo ; le but de l'association est de sortir la philosophie du cadre de l'enseignement et d'ouvrir des débats plus larges par exemple sur la liberté. Site : \url{http://depidalate.free.fr/}
%\paragraph{Maison des Jeux} Tous les mercredis midi au TU, tu pourras découvrir et participer à des jeux de société et jeux en bois.


%\section{Trouver un logement}\trad
\section{Find a home}
%\paragraph{Cité universitaires} En tant que doctorant tu es normalement prioritaire pour obtenir un logement du CROUS. Le mieux est de se rendre directement  à l'accueil du CROUS (2 bd Guy Mollet arrêt de TRAM Faculté) et se référer au site internet du CROUS www.crous-nantes.fr. ATTENTION, il y a beaucoup de demandes, il est donc préférable de s'y prendre plusieurs mois à l'avance (au mois d'Avril pour Septembre) !
\paragraph{Campus}
As a PhD student, you are normally a priority for housing CROUS.
The best is to go directly to the CROUS house(2 boulevardd Guy Mollet, near the TRAM stop Faculty) and take a look at the CROUS website www.crous-nantes.fr.
WARNING, there are many applicatants, so we advise you to ensure your place several months in advance (in the month of April to September)!

%\paragraph{Colocation} Si vivre à plusieurs te tente, tu peux trouver sur www.appartager.fr plein d'annonces gratuites pour partager les frais et faire des rencontres très rapidement.

\paragraph{Roomates}
You can find lots of ads on www.appartager.fr to share costs. As a bonus, you'll meet new people very quickly.

%\paragraph{Location d'appartement} Pour trouver un appartement à louer, il y a deux possibilités : les agences immobilières (ATTENTION, il faut généralement payer des frais d'agence pouvant s'élever à 1 mois de loyer), les annonces de particuliers (leboncoin.fr, de particulier à particulier pap.fr et/ou le CROUS). Penses également à acheter le supplément immobilier du journal Ouest France publié tous les samedi !

\paragraph{Renting an Apartment}
To find an apartment, there are two possibilities: realestate agencies (WARNING: you'll generally end up paying finders fees of up to one month's rent), or advertisements by owner (leboncoin.fr "of particular to particular", pap.fr and / or CROUS).

%Pour le lieu de ton logement tu feras face au dilemme classique :
%\begin{itemize}
%  \item Habiter au centre ville, près de toutes les animations étudiantes, facilement accessible par le tram pour te rendre au boulot. Mais tu risques d'avoir un petit logement cher avec des difficultés pour garer ta voiture.
%  \item Habiter une petite maison pas très chère en banlieue au milieu de la nature avec jardin et barbecue mais tu devras investir dans une voiture et faire une croix sur le troisième verre pour rentrer chez toi...
%  \item Dormir dans ton bureau (tu remarqueras comme la moquette est confortable... et propre !!)
%\end{itemize}

As to where you want to live, you will face the classic dilemma. Your options:
\begin{itemize}
  \item Live in the city center: convenient since it is close to all the students activities, with easy tram access to get to work. But you may have a small flat with high rent and difficulties parking your car if you have one.
  \item Live in a small house in the suburbs: much cheaper, but you will have to invest in a car to go out \dots
  \item Sleep in your office :as you will soon find out, the carpet is clean and has a surprisingly high thread count! 
\end{itemize}

%Lors de ta première année de thèse, tu peux avoir droit aux APL (Aides Personnalisées au Logement) puisqu'elles se basent sur les revenus que tu as obtenus l'année précédente pour calculer ton aide (penses à valoriser ton statut d'étudiant). Se renseigner sur le site \url{www.caf.fr} de la Caisse d'allocation familiale qui délivre cette aide personnalisée. En tout cas, mieux vaut faire la demande, ça ne mange pas de pain.
During your first year PhD student, you may be eligible for APL (Custom Aids for Housing) since your eligability is based on your income of the previous year.
Learn more on the website \url{www.caf.fr}.

%\section{Manger près de Centrale}\trad
\section{Places to eat near Central}

%\subsection{Les restaurants universitaires (ou « RU »)}
\subsection{University restaurants (or "RU")}
%Les « restos U » auxquels tu peux avoir facilement accès à pied depuis Centrale se comptent au nombre de trois : le Tertre, le Rubis et la Cafétéria du pôle étudiant. Il en existe d'autres à Nantes et tu trouveras toutes les informations les concernant sur le site du CROUS de Nantes (\url{http://www.crous-nantes.fr}). Le tarif d'un plateau-repas au RU se situe aux alentours des 3 euros (tarif 2011 – 2012).
%Le paiement s'effectue exclusivement grâce à ta carte d'étudiant Pass'Sup, qui fait aussi office de carte Moneo.

%Tu peux créditer ta carte Pass' Sup dans n'importe quels bureaux de poste, banques, cabines téléphoniques, halls d'accueil de restaux U, etc... arborant le logo « Moneo », ou encore en ligne sur \url{www.moneo.net} !!

There are three "restos U" that you can easily access by foot from Centrale: Tertre, Ruby and the student division Cafeteria. There are others in Nantes and you will find all the information about them on the site CROUS Nantes (\url{http://www.crous-nantes.fr}). The price of a meal in the RU is around 3 euros (2011 - 2012).

Payment can only be made through your student card Pass'Sup, which also serves as Moneo card (a pre-charged account for small transactions).

You can add funds to your card Pass' Sup in any post office, some banks, phone booths, and the reception halls of restaux U, etc \dots bearing the logo "Moneo" or online at \url{www.moneo.net}!

%\subsubsection{Le Tertre}
\subsubsection{The Tertre}
%Le Tertre se situe juste en face de l'arrêt de tramway Facultés. Il comporte deux étages et permet un choix relativement varié de cuisines.
%On y trouve au rez-de-chaussée une sandwicherie ainsi qu'un coin café et à l'étage plusieurs pôles thématiques (cuisine du monde, poisson, grill, pizza, etc...).

%Ouverture : Le midi (11h00-13h30) du lundi au vendredi.

The Tertre is located just opposite the tram stop "Facultés". It has two floors and provides a relatively varied menu.
On the ground floor, you can found a sandwich and a coffee corner and upstairs several themed stands (world cuisine, fish, grill, pizza, etc \dots).

Oprating Hours: afternoons (11:30 to 1:30 p.m.) Monday through Friday.
%\subsubsection{Le Rubis}
\subsubsection{The Rubis}

%L'intérêt principal de ce restaurant est qu'il est en général moins pris d'assaut que son homologue le Tertre et qu'en plus il est ouvert le soir (avec une sympathique petite musique de fond bien agréable pour digérer toutes les formules et équations de la journée).
%Ainsi, tu pourras travailler ta thèse jusqu'à 19h00, manger au RU et enfin revenir pour poursuivre ton travail (jusqu'à la fermeture de l'école à 22h00 pile)

%Ouverture : Le midi (11h00-13h30) et le soir (18h30-20h00) du lundi au samedi midi.

The main draw of this restaurant is that is often less crowded than the Tertre. It is also open in the evening.

Operating Hours: lunch (11h00-13h30) and dinner (18h30-20h00) from Monday to Saturday.


%\subsection{La Cafétéria du pôle étudiant}
\subsection {The Student Cafeteria}
%La cafétéria du pôle étudiant se situe entre les Facs de Droit et de Lettres.
%Elle est ouverte en continu de 8h30 à 17h30 et l'ambiance est assez "chaleureuse" (ambiance Lettre et Droit quoi...).
%On y sert des repas "chauds" en continu (paninis, tapas, etc...) et tu y trouveras aussi crudités et viennoiseries.

%Ouverture : Du lundi au Vendredi de 9h30 à 21h00.

The student center cafeteria is located between Law and Letter university buildings.
It is open continuously from 8:30 to 17:30 and is very confortable.
It serves hot meals continuously (paninis, tapas, etc \dots) as well as salads and pastries.

Operating Hours: Monday to Friday from 9:30 to 21:00.

%\subsection{Le bâtiment L (la K-fet' de l'école)}

%Il fut un temps pas si lointain où le bâtiment L de l'école abritait une cafétéria permettant aux élèves et au personnel de s'y restaurer sans quitter le campus.
%Cette K-fet' ayant fermé ses portes pour des raisons administratives en 2012, il n'est plus possible de s'y rendre le midi.
%Cependant, la rumeur court qu'elle pourrait renaître de ses cendres ; affaire à suivre...
%Pas traduit car pas intéressant

%\subsection{La baraque à frites}
%Quand les RU sont fermés ou quand le temps te manque, tu trouveras en général une camionnette-snack garée au niveau du rond point de l'école.
%Double-américain avec surplus de ketchup et frites au menu !!

%Ouverture : Le midi du lundi au vendredi.

\subsection{The French Fries}
When the RUs are closed, or when you are in a rush, you can usually find a snack van parked in front of the school. The fare is mostly burgers, kebab, and fries.
Double-American with extra ketchup and fries!

Operating Hours: 11:00-13:30 (ish), from Monday to Friday.

%\subsection{Le Fac' Food}

%Le Fac' Food est, tu l'as deviné, une sandwicherie. Elle est située en face de l'école et tu y trouveras sandwichs, burgers en tous genres, paninis, américains, etc...

%Ouverture : Du lundi au vendredi de 8h00 à 15h00.

\subsection{The Fac 'Food}

The Fac 'Food is a sandwich shop located in front of the school near the entry roundabout. There you will find sandwiches, burgers of all kinds, paninis, American cuisine, etc \dots

Operating Hours: Monday to Friday 8:00 to 15:00.

%\subsection{Pizzeria-brasserie « Le Café Gourmand »}

%Cette brasserie jouit d'une bonne réputation sur le net et elle est pourvue d'une grande terrasse-jardin à l'arrière.
%Formule "Pizza - Café Gourmand" à 14 euros.

%Ouverture : Le midi du lundi au vendredi et le jeudi soir (pizza à emporter)



%\section{Aide aux étudiants étrangers}\trad
%Tu es un chercheur étranger à Nantes ?
%L'association « Chercheurs Étrangers à Nantes » constitue un guichet unique en matière d'information et d'accueil.
%Parmi les services proposés, on retrouve :
%\begin{itemize}
%  \item L'aide à la préparation du séjour,
%  \item Des services personnalisés pour ton arrivée (information sur la vie quotidienne, transport, plans, scolarisation des enfants...),
%  \item Un appui aux démarches administratives de séjour (constitution et dépôt du dossier de demande de titre de séjour, informations pratiques sur les taxes, salaires, assurances, visite médicale, banque, etc...).
%  \item Des cours de français en tant que langue étrangère (à faible coût), niveaux débutant et avancé
%\end{itemize}

\section{Help for foreign students}
Are you a foreign researcher in Nantes?
The public association Maison de Cherchers Etrangers aims to assist and inform foreign researchers .
Among the services offered:
\begin{itemize}
   \item personalized services for your arrival (information on daily life, transport, maps, children's schooling \dots)
   \item Support in administrative tasks (filling and submitting the application for a resident permit, practical information on taxes, wages, insurance, medical, banking, etc \dots).
   \item French as a foreign language classes for beginner and advanced levels
\end{itemize}
%Par ailleurs, tu as envies de participer, de créer ou de t'engager ?
%Les associations à Nantes sont à ton écoute !
%Tu trouveras des associations impliquées dans plusieurs secteurs d'activité.
%N'hésite pas à visiter le site internet de l'Université de Nantes (rubrique vie associative) pour plus d'informations.


%\subsection{Association de regroupement culturel}
%Les associations des étudiants étrangers ont pour but d'être un repère pour un étudiant étranger à Nantes et de promouvoir leurs cultures (danse, alimentation, musique, ...).
%On retrouve l'association des étudiants libanais à Nantes \textbf{ANADYL} (\texttt{asso\_anadyl@yahoo.fr}), l'association des étudiants guinéens à Nantes \textbf{AGEN} (\texttt{agen.nantes@yahoo.com}), l'association des étudiants vietnamiens à Nantes \textbf{AEVN} (\texttt{contact@aevn.fr}), l'association des étudiants gabonais à Nantes \textbf{GaboNantes} (\texttt{gabo\_nantes2@yahoo.fr}), l'association des étudiants italiens à Nantes \textbf{Issimo.it} (\texttt{issimo.nantes@gmail.com}) et l'association des étudiants Roumains à Nantes \textbf{Roumanie d'ici} (\texttt{roumaniedici@yahoo.fr}).
%N'hésite pas à te faire connaître auprès de ces associations.
%Et d'ailleurs si tu es Français n'hésite pas non plus à te faire passer pour un ERASMUS car en France, il n'y a pas d'associations d'étudiants français !

\subsection{Cultural Association}
%TO DO: dépasse dans la marge à droite, à corriger
You can find several associations of foreign students, all promoting the culture of their respective countries:
\begin{description}
    \item[ANADYL] the association of Lebanese students in Nantes \texttt{associated\_anadyl@yahoo.fr})
    \item[AGEN] the Association of Guinean students in Nantes (\texttt{agen.nantes@yahoo.com})
    \item[AEVN]  the Association of Vietnamese students in Nantes (\texttt{contact@aevn.fr)}
    \item[GaboNantes] the Association of Gabonese students in Nantes (\texttt{gabo\_nantes2@yahoo.fr})
    \item[Issimo.it] the association of Italian students in Nantes (\texttt{issimo.nantes@gmail.com})
    \item[Roumanie d'ici] the Association of Romanian students in Nantes (\texttt{roumaniedici@yahoo.fr}).
\end{description}
Do not hesitate to contact these associations, to meet new people and share your culture!

