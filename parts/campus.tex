
\section{Tout sur l'École Centrale de Nantes}\trad
\subsection{L'école en quelques chiffres}\trad
L'école a été fondée il y a plus de 90 ans.
D'abord appelée l'IPO (Institut Polytechnique de l'Ouest depuis 1919), puis l'ENSM (École Nationale de Mécanique à partir de 1947), elle est rattachée au groupe des Écoles Centrales en 1991.
Elle était initialement basée en centre ville, elle est arrivée sur le campus du Tertre en 1977.
\todo{Vérifier ces chiffres}
\paragraph{Les gens}
\begin{itemize}
  \item 1850 étudiants (1200 élèves-ingénieurs, 250 élèves-ingénieurs en formation continue et par apprentissage, 200 étudiants de Master, 200 doctorants),
  \item 550 chercheurs, enseignants-chercheurs et personnels de recherche,
  \item 150 personnels administratifs et techniques.
\end{itemize}
\paragraph{Le Lieu}
\begin{itemize}
  \item 40000 m\textsuperscript{2} de locaux,
  \item 1 campus de 16 ha,
  \item 5 laboratoires labellisés CNRS.
\end{itemize}
\paragraph{Centrale Nantes en France et dans le monde}
\begin{itemize}
  \item 11\textsuperscript{e} meilleure École d'Ingénieurs de France (classement 2010),
  \item 115 universités partenaires dans 40 pays.
\end{itemize}

\subsection{Les grandes têtes de l'école}\trad
Voici une petite liste (non exhaustive) de personnalités importantes du campus, que vous serez amenés à croiser durant vos 3 ans de thèse.
\todo{Compléter les noms et vérifier les nouveaux intitulés des fonctions}

\newcommand{\largcol}{0.29\textwidth}

\newcommand{\pers}[4]{%
\begin{tabular}{m{\largcol}}%
\ifthenelse{\equal{#1}{}}{\todo{Photo}}{}%
%\ifthenelse{\equal{#1}{}}{\todo{Photo}}{\includegraphics[width=2.5cm]{#1}}%
%\includegraphics{#1}
\\ \textbf{#2} \\ \textit{(#3)} \\ #4
\end{tabular}}

\noindent
\begin{tabular}{*{3}{p{\largcol}}}
  \pers{}{}{Directeur de l'école}{C'est notre chef à tous. On lui doit allégeance, même si en pratique, on a rarement affaire à lui.}&
  \pers{}{}{Directeur des études}{Comme son statut l'indique, il est davantage en lien avec les élèves. Amis moniteurs, si vous avez des problèmes avec vos enseignements, c'est peut-être à sa porte qu'il faut aller frapper.}&
  \pers{}{}{Directeur de la recherche}{C'est notre boss à nous. Il gère la politique de recherche de l'école (les projets, l'argent, les diverses activités, ...) et peut s'occuper de nous en cas de [gros] problèmes avec notre thèse.}\\
  \pers{}{}{Secrétaire général}{Il s'occupe des grandes décisions administratives qui nous concernent (oui, la paperasse de l'inscription, c'est en partie pour son service...). Véritable encyclopédie des textes officiels, il pourra répondre aux questions auxquelles tout autre secrétaire aura jeté l'éponge.}&
  \pers{}{}{Directeur de la communication}{Lui et son équipe gèrent la diffusion des informations et l'événementiel sur le campus. Son travail dépend notamment de votre participation, alors n'hésitez pas à l'informer de tout ce qu'il se passe dans votre labo.}&
  \pers{}{}{Directeur adjoint}{C'est l'homme de terrain. Il met à exécution les grandes décisions « concrètes » de l'administration. Vous recevrez notamment pas mal de mails durant l'année sur d'éventuels travaux, changement d'organisation, etc. de sa part.}
\end{tabular}

\subsection{Les étudiants}\trad
L'école renferme une population d'étudiants assez hétéroclite.
Il y a :
\paragraph{Les élèves-ingénieurs (EI1, EI2 et EI3)} Ils sont admis à l'école sur concours après une classe prépa (bac+2) et constituent la majorité des étudiants du campus et sont présents pour 3 ans. Très actifs sur le plan associatif, ils animeront le campus tout au long de l'année. Les premières années vivent dans la résidence près du campus. Depuis peu, certains élèves ingénieurs travaillent en alternance au même titre que les ITII (voir plus bas).
\paragraph{Les Masters (M1 et M2)} Originaires d'un parcours universitaire tiers, les masters partagent souvent leurs cours avec les élèves ingénieurs. On peut notamment citer les Master Erasmus-Mundus attirant des étudiants du monde entier !
\paragraph{Les ingénieurs par alternance ITII Pays de la Loire (1ère, 2ème et 3ème année)} Il s'agit d'étudiants en formation par alternance (2 semaines en cours, 2 semaines en entreprise). L'ITII Pays de la Loire est un organisme de formation extérieur à l'École Centrale Nantes, cependant les cours se font sur le campus. Officiellement, ce ne sont pas des étudiants. Dans la pratique, ils sont exactement comme les autres, à la différence qu'ils ne sont présents que 2 semaines sur 4.
\paragraph{Les doctorants (D1, D2 et D3)} C'est toi.

\subsection{Les labos de l'école}\trad
Le campus de l'École Centrale comprend 5 labos (voir le plan du campus) :
\paragraph{Le GeM} De son vrai nom « Institut de Recherche en Génie civil et Mécanique ». Le Gem est un labo installé sur 3 sites : l'École Centrale, la fac de science et l'IUT de St Nazaire. On y fait de la simulation mécanique, du génie civil, des études de matériaux, de la dynamique rapide (crash-test)... Parmi tous les thèmes de recherche (et il y en a beaucoup), deux ressortent : La méthode X-FEM (méthode des éléments finis enrichis) et la PGD (méthode de décomposition de variables pour résoudre des problèmes beaucoup plus rapidement). D'autre part, un domaine d'application particulièrement en vogue ces temps ci est celui des matériaux composites.
\paragraph{Le LHEEA} (« Laboratoire de recherche en Hydrodynamique, Énergétique et Environnement Atmosphérique ») C'est sans doute le laboratoire qui possède le matériel expérimental le plus impressionnant du campus : un bassin de carène (pour tracter des maquettes de bateaux), un bassin de houle (pour faire des vagues), une soufflerie... Le LHEEA est morcelé en plusieurs équipes dont les principaux thèmes sont l'hydrodynamique et génie océanique (projet « SEAREV » de récupération d'énergie des vagues), l'énergétique des moteurs à combustion interne, l'étude de la dynamique de l'atmosphère en milieu urbain.
\paragraph{L'IRCCyN} (« Institut de Recherche en Communications et Cybernétique de Nantes ») Principalement installé dans le bâtiment S (le bâtiment relativement moderne, de forme ovale), l'IRCCyN concentre douze thématiques de recherche très variées. On y travaille par exemple sur la commande, la robotique, la conception, la fabrication additive, l'optimisation multidisciplinaire, les systèmes temps réels et même... la psychologie ! L'IRCCyN est réparti sur 4 sites de Nantes : l'École Centrale, l'École des Mines, l'École Polytechnique et l'IUT.
\paragraph{Le CERMA} (Centre de Recherche Méthodologique d'Architecture). Chapeauté depuis peu par l'École Centrale, le CERMA est avant tout un laboratoire dépendant de l'École Nationale Supérieure d'Architecture de Nantes. On y travaille notamment sur toute la physique liée l'architecture et l'environnement urbain.
\paragraph{Le laboratoire de mathématique Jean Leray} On n'en entend pas trop parler, mais il est là, caché dans une partie du bâtiment E ! En fait, c'est parce que l'essentiel se trouve à la fac de sciences et seul un petit bout est à l'École Centrale Nantes. La recherche à Centrale se concentre sur l'étude des équations aux dérivées partielles non-linéaire avec des domaines d'application pouvant aller des problèmes dynamique de populations à l'étude des changement d'état de matériaux supraconducteurs.

\subsection{Plan du site de l’École Centrale Nantes}\trad
\todo{Inclure le plan}