\chapter{Visiter Nantes}\trad

Capitale de la région des Pays de la Loire, Nantes est aujourd'hui la 6ème ville de France et se situe au centre d'une agglomération de plus de 580 000 habitants ! Mais, pas de panique, 35 \% de la population a moins de 25 ans ! Ça en fait du monde à rencontrer ! Et oui, la douceur de vivre à Nantes a attiré un afflux record de citadins et continue de séduire bien des touristes.
Dans cette partie, vous retrouverez plusieurs plans de la ville et des environs, avec pleines de bonnes idées pour découvrir et s'amuser !

\section{Pour bien manger}\trad

\paragraph{Les restaurants} le coin resto/bars le plus fréquenté est le quartier Bouffay (situé entre les repères 2 et 9). Tu y trouveras une foultitude de crêperies, brasseries, japonais et pubs.
\paragraph{Faire ses courses} A Nantes, n'essaie pas de chercher un Cora ou un Simply, tu devras principalement choisir entre :
Leclerc (au Centre commercial Atlantis à Saint Herblain, entre autres)
Carrefour (au centre commercial de Beaulieu) ou les petits carrefour contact et carrefour city
Auchan (route de Vannes à Saint Herblain)
Monoprix
\paragraph{Faire son marché} Pour celles/ceux qui aiment les légumes/fruits de saison, un marché se tient
tous les samedis matin (jusqu'à 13h, heureusement pour les lève-tards) place Gloriette, en face de l'arrêt Médiathèque (M1).
tous les matins de la semaine à Talensac
un petit marché bio le mercredi matin de 8h à 13h à Commerce, sur le square derrière la Fnac.

\section{Point Sport et culture}\trad
\paragraph{Piscine}
A Nantes, 6 piscines sont disponibles. Tu trouveras dans le centre la piscine Léo Lagrange équipée d'un bassin sportif de 50 mètres. A 3 minutes de Centrale, la piscine du Petit Port toute neuve te permettra de te défouler quand tu pètes un plomb sur ton code Matlab ! L'entrée est d'environ 3 euros mais il existe également des cartes à points qui permettent de bénéficier de tarifs préférentiels et des réductions étudiantes. Plus d'infos sur le site de la ville : www.nantes.fr/detente/piscines-nantaises/
\paragraph{Parcs}
Le parc de Procé       est le parc le plus apprécié en ville ; on peut le regagner à pied en suivant une rivière que l'on pourra rejoindre à partir de la rue de Lamoricière (proche arrêt Chantier Naval). Le parc est également idéal pour ceux qui pratiquent le footing ; en effet, après avoir traversé le parc, on peut passer des portes grillagées qui mènent à une zone plus champêtre (où se situe la piscine des Dervallières, un terrain de foot et des terrains de tennis). Au final, on peut faire un parcours en boucle d'environ 10 km à partir de l'entrée du parc.
A la belle saison, tu peux aussi profiter de la fraicheur du Jardin des Plantes        ; la collection impressionnante de fleurs et l'aménagement paysager avec ses fontaines t'en mettent plein la vue ! Idéal pour venir bouquiner tranquille également... ou tout simplement attendre ton train.
\paragraph{Cinémas}
Sur la place du commerce, à côté de la Fnac, se tient un cinéma Gaumont Pathé. Pratique lorsqu'on habite au centre-ville et que l'on veut enchaîner sur le restaurant d'à côté, mais les prix restent élevés (7-8 euros en étudiant) !
Au centre commercial Atlantis, à l'ouest de Nantes (accès : ligne 1 ou voiture), il existe deux complexes, Pathé Atlantis et UGC Atlantis. Ils proposent des places en -26 ans aux alentours de 4 euros pour toutes les séances.

\section{Où sortir à Nantes ?}\trad
Pour la fête, on va te faire confiance. Sinon, voici trois coups de pouce :
\begin{itemize}
  \item Le magazine « Pulsomatic » recense tous les spectacles et concerts de Nantes et sa région ; tu peux trouver la version papier à l'accueil des RU, du CROUS ou d'autres établissements étudiants, sinon rend-toi directement sur le site \url{www.pulsomatic.com}.
  \item Le site www.onvasortir.com te met en relation avec des personnes de ta ville/région pour rencontrer et s'amuser autour de thèmes communs.
  \item Va profiter des transats, boire un verre ou danser au hangar à bananes : le long du quai des Antilles, les anciens entrepôts où murissaient les bananes importées de Guinée ou de Guadeloupe ont été aménagés pour accueillir galeries d'art, restaurants, une boite de nuit (le LC) et des bars à thèmes. Le long du quai, tu peux également voir les 18 anneaux de Buren, que tu apprécieras plus à la nuit tombée, lorsqu'ils s'éclairent. Et la grande grue jaune à la pointe de l'île s'appelle Titan !
\end{itemize}

\section{A faire absolument !}\trad
\begin{itemize}
  \item Se balader dans le jardin japonais de l'île de Versailles au cadre très romantique (attention, il y a du couple au m\textsuperscript{2}...). Pour les sportifs, on peut également louer des canoës-kayak. Durant le mois de Juin, tu iras écouter des concerts de jazz gratuits dans le cadre du festival « Les Rendez-Vous de l'Erdre ».
  \item Faire le tour des remparts du château des ducs de Bretagne, c'est gratuit et tu auras un beau panorama : attention, ça vente ! Le musée du château est aussi très sympa. L'histoire de Nantes vous y sera contée avec les derniers outils intéractifs du moment. Idéal pour s'occuper à Nantes un jour de pluie (tellement rare dans cette région..) ou encore pour trouver un coin au frais les jours de canicule.
  \item Prendre le navibus pour Trentemoult (entre les arrêts Chantier Naval et Gare maritime, de la ligne de métro 1) ; ancien village de pêcheur, Trentemoult est maintenant un quartier branché, car bien restauré par les nouveaux habitants. Il donne une agréable sensation de dépaysement, à quelques minutes seulement du centre-ville. Le navibus est considéré comme une navette classique du réseau tan, ce qui implique que le prix du ticket n'est pas plus cher que pour le reste du réseau (et que la carte d'abonnement suffit ! si vous l'avez). Ne cherches pas ailleurs, les seuls bars/restaurants sont ceux que tu vois en arrivant de la navette.
  \item Visiter la galerie des machines et regarder marcher l'Eléphant ; et oui, un éléphant vit sur l'île de Nantes et vous le trouverez sagement parqué dans son hangar à droite à la sortie du pont Anne de Bretagne. Il est le premier de toute une série de machines articulées et de projets fantasmagoriques, qui font parti du projet de réaménagement de l'île de Nantes. Dans la galerie des machines, des animateurs vous expliqueront la construction de l'Eléphant et d'autres machines et vous pourrez même en tester certaines ! Ça fait rêver... Pour connaître les horaires ou les tarifs, se rendre sur le site : \url{www.lesmachines-nantes.fr}
  \item Le hangar à bananes
  \item Se balader sur les berges de l'île de Nantes, observer les vieux gréements qui y séjournent ou qui sont juste de passage.
\end{itemize}

\section{Événements à ne pas louper !}\trad
\begin{itemize}
  \item Ne pas rater le passage de ROYAL DELUXE avec ses géants qui sillonnent le centre-ville
  \item Festival de musique classique : les « folles journées de Nantes » aux mois de janvier-février
  \item Les journées Escapades (un week-end entre avril et mai), durant lesquelles vous sont proposées des initiations à des sports de plein air et des démonstrations (les sports au programme : kite-surf, catamaran, tir-à-l'arc, speed-sail, équitation, char-à-voile, beach soccer, beach volley) ; retrouvez le programme sur le site de la région Loire-Atlantique ou dans le magazine gratuit de la région Loire-Atlantique que vous recevez normalement par courrier !
\end{itemize}

\section{Les plus :}\trad
\begin{itemize}
  \item Pense à l'abonnement de tram, remboursé en moitié par ton employeur (par l'école centrale si tu es embauché par celle-ci) !
  \item Pense à venir à l'école en vélo dès les beaux jours (en voiture, c'est la galère !). Des locations de vélos sont proposées aux étudiants pour \EUR{45} par an (Vélocampus). Et le réseau Bicloo s'étend jusqu'à l'Ecole Centrale. Les bords de l'Erdre pour aller au boulot c'est le pied !
  \item Et pour ceux qui veulent connaitre toutes les facettes de Nantes et aiment les jeux de pistes, vous trouverez votre bonheur avec les cistes : \url{http://www.cistes.net/}
\end{itemize}



